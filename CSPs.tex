%!TEX root=main.tex
\section{Constraint Satisfaction Probleme}
Eine Variablenbelegung unter bestimmten Bedingungen zu finden ist ein fundamentales Problem. \emph{Constraint Satisfaction Probleme (CSP)} formalisieren diese Problemstellung.
\begin{definition}[Constraint Satisfaction Problem]
	Ein \emph{Constraint Satisfaction Problem (CSP)} besteht aus $n$ Variablen $X_i$ mit zugehörigen \emph{Domains} $D_i, x_i\in D_i$ und $K$ \emph{Constraints} $C_k$.
	Jedes Constraint trifft eine Aussage über die Belegung einer Teilmenge aller Constraints. Genauer ist ein Constraint $C_k$ ein Tupel $C_k=(I_k,c_k)$ wobei $I_k\subseteq\simpleset{1,\ldots,n}$ die Teilmenge der betroffenen Variablen ist. Die Abbildung $c_k:D_{I_k}\rightarrow \simpleset{0,1}$ bestimmt, ob die Konfiguration der $x_{I_k}\in D_{I_k}$ zulässig ist.

	Das Ziel ist, eine Konfiguration $X=(X_1,\ldots,X_n)$ aller Variablen zu finden, sodass alle Constraints erfüllt sind.
\end{definition}

Ein CSP lässt sich als \emph{Constraint Graph} darstellen. Dies ist ein bipartiter Graph, in dem Variablen als Kreise und Constraints als Kästen dargestellt werden. Ein Beispiel ist der nachfolgende Constraintgraph.
\noindent
\begin{center}
	\begin{tikzpicture}[auto,node distance=2.5cm,semithick, baseline=1em]
	  \tikzstyle{every state}=[fill=white,circle,draw=black,text=black,align=center]

	  \node[state] (W) {W};
	  \node[state] (N) [below left of=W] {N};
	  \node[state] (S) [below right of=W] {S};
	  \node[state] (T) [above right=0.2cm and 1.5cm of W] {T};

	  \path (W) edge node[rectangle, fill=black, label={above left:$c_2$},below=-3pt]{} (N)
	  (N) edge node[rectangle, fill=black, label={below:$c_4$},below=-3pt]{} (S)
	  (W) edge node[rectangle, fill=black, label={above right:$c_3$},below=-3pt]{} (S)
	  node[rectangle, fill=black, label={above:$c_1$},above left=0.45cm and 0.45cm of W]{} edge (W)
	  node[below=1.4cm of W] {} edge (W);
	\end{tikzpicture}
\end{center}
\noindent
Das zugehörige CSP hat 4 Variablen und ebenfalls 4 Constaints. Man sieht, dass die Constraints unterschiedliche Anzahlen an Variablen beeinflussen können.
Es gilt $I_1=\simpleset{W},I_2=\simpleset{W,N},I_3=\simpleset{W,S}$ und $I_4=\simpleset{N,W,S}$. Auch möglich ist, dass eine Variable von keinem Constraint beeinflusst wird, wie an $T$ zu sehen.
\begin{itemize}
	\item $C_1$ ist ein sogenanntes \emph{unäres Constraint}, da $|I_1|=1$.
	Ein Beispiel wäre $W\neq \text{rot}$.
	\item Die Constraints $C_2,C_3$ sind \emph{binäre Constraints}, da $|I_2|=|I_3|=2$. Ein Beispiel wäre $W\neq N$.
	\item Und das Constraint $I_4$ ist ein \emph{Constraint höherer Ordnung}, da $|I_4|>2$.
\end{itemize}

Sind alle Domainräume endlich und haben die gleiche Größe $d=|D_i|$, dann ist der Aufwand das CSP zu lösen in $\mathcal O(d^n)$.

\subsection{Lösen eines diskreten CSP}
Der \enqo{brute force} Lösungsalgorithmus für ein CSP ist \emph{sequential assignment}. Der Algorithmus weist Variablen Werte zu, solange dies nicht den Constraints widerspricht. Implementiert wird dies als eine Tiefensuche in einem Baum mit branching factor $(n-l)d$ bei Tiefe $l$. Der Baum hat also $n!d^n$ Blätter!

Eine erste Verbesserung dieses Algorithmus ist \emph{Backtracking Sequential Assignment}. Die entscheidende Erkenntnis ist, dass es für die Vollständigkeit des Algorithmus keine Rolle spielt in welcher Reihenfolge die Variablen gesetzt werden.
Insbesondere kann man also eine Reihenfolge festlegen und damit den branching factor des Baums drastisch reduzieren. In jeder Tiefe ist der der branching factor $d$, was nur noch $d^n$ Blätter zu Folge hat.
Hiermit erreicht man den oben angesprochenen Zeitaufwand von $\mathcal O(d^n)$.

08/13
\begin{lstlisting}
function backtrackingSearch(csp)
	return recBacktracking({}, csp)
end

function recBacktracking(assignment, csp)
	if assignment.isComplete then return assignment
	var = selectUnassignedVariable(variables(csp), assignment, csp)
	for value in orderedDomainValues(var, assignment, csp) do
		if value consistent with assignment given constraints(csp) then
			add [var = value] to assignment
			result = recursiveBacktracking(assignment, csp)
			if result != failure then return result
			remove [var = value] from assignment
		end
	end
	return failure
end
\end{lstlisting}

\subsection{Möglichkeiten der Verbesserung}
Mithilfe von simplen Heuristiken kann man die Laufzeit des Lösungsalgorithmus stark verringern. Dabei gibt es mehrere Angriffspunkte zur Optimierung, wir betrachten zunächst welche Variable am besten als nächstes zur Belegung gewählt werden sollte sowie die Reihenfolge der Wertebelegungen.
\subsubsection{Variablenreihenfolge}
Mit der \emph{Minimum remaining Values (MRV)}-Heuristik kann man die Breite des Baums stark verringern. Wir wählen die Variable, der als nächstes ein Wert zugewiesen werden soll so, dass sie die wenigsten möglichen Wertebelegungen hat. So ist der branching factor des Baums zu Beginn klein.

Hier können natürlich Fälle auftreten, bei denen diese Entscheidung nicht eindeutig ist. Wir führen die \emph{Degree Heuristic} als Tie Breaker ein. Hier wird immer die Variable als nächstes gewählt, die die meisten Constrains auf noch unbelegte Variablen hat. Damit werden gleichzeitig die möglichen Wertebelegungen der bisher unbelegten Variablen verringert.

\subsubsection{Wertereihenfolge}
Sind bei einer Variablenbelegung mehrere Werte möglich wählt man mit der \emph{Least Constraining Value}-Heuristik immer den Wert, der noch die meisten Wertebelegungen bei den noch unbelegten Variablen zulässt.

\subsubsection{Inferenz}
Mit den bisher besprochenen Optimierungen ist nicht alles abgedeckt. Durch Anwenden von Inferenz können wir frühzeitig feststellen ob eine Lösung überhaupt noch möglich ist, was natürlich bedeutet dass wir das Backtracking früher beginnen können.
\paragraph{Constraint Propagation}
Nach jeder Wertezuweisung können die möglichen Belegungen der übrigen Variablen berechnet werden. Das heißt nach jeder Zuweisung werden aus den Domains $D_i$ entsprechend den Constraints alle nicht mehr möglichen Werte entfernt. Dies kann rekursiv geschehen, da das Löschen eines Werts wiederum die Domains beeinflussen kann.
Diese Denkweise ist \emph{Inferenz}. Das angewandte Verfahren heißt \emph{Constraint Propagation}, da wir implizite Constraints propagieren und so die Domains verkleinern.

Für binäre Constraints wird Constraint Propagation auch \emph{Arc Consistency} genannt. 
\begin{definition}[Konsistenz bei Arc Consistency]
	Ein Constraint $X\rightarrow Y$ \emph{konsistent}, wenn für jeden möglichen Wert $x$ von $X$ ein erlaubter Wert $y$ existiert.
\end{definition}
Nach dieser Regel werden Werte aus den Domains entfernt.

Betrachten wir ein kleines Beispiel. Die nachfolgenden Variablen haben alle den gleichen Domainraum $D=\simpleset{\text{\color{red}rot},\text{\color{blue}blau},\text{\color{green}grün}}$. Alle Constraints fordern, wie im Constraintgraph eingetragen, Ungleichheit der Variablen. Es handelt sich hier also um ein Färbungsproblem. Die Domains wurden in die Nähe der Variablen eingetragen.
\noindent
\begin{center}
	\begin{tikzpicture}[auto,node distance=2.5cm,semithick, baseline=1em]
	  \tikzstyle{every state}=[fill=white,circle,draw=black,text=black,align=center]
	  \tikzstyle{constraint}=[rectangle, fill=black, below=-3pt]

	  \node[state] (N) {N};
	  \node[state] (W) [below left=0.7cm and 1.2cm of N] {W};
	  \node[state] (S) [below of=N] {S};
	  \node[state] (T) [below right=0.7cm and 1.2cm of N] {T};

	  \path 
	  (N) edge node[constraint, label={above left:$W\neq N$},below=-3pt]{} (W)
	  edge node[constraint, label={above right:$N\neq T$}] {} (T)
	  

	  (S) edge node[constraint, label={below right:$S\neq T$}] {} (T)
	  edge node[constraint, label={left:$N\neq S$}]{} (N)
	  edge node[constraint, label={below left:$W\neq S$}]{} (W)
	  ;

	  \node[constraint] [fill=red,above left=0.1cm and 0.1cm of N] {};
	  \node[constraint] [fill=green,above=0.1cm of N] {};
	  \node[constraint] [fill=blue,above right=0.1cm and 0.1cm of N] {};

	  \node[constraint] [fill=red,above left=0.1cm and 0.1cm of W] {};
	  \node[constraint] [fill=green,left=0.1cm of W] {};
	  \node[constraint] [fill=blue,below left=0.1cm and 0.1cm of W] {};

	  \node[constraint] [fill=red,below left=0.1cm and 0.1cm of S] {};
	  \node[constraint] [fill=green,below=0.1cm of S] {};
	  \node[constraint] [fill=blue,below right=0.1cm and 0.1cm of S] {};

	  \node[constraint] [fill=red,above right=0.1cm and 0.1cm of T] {};
	  \node[constraint] [fill=green,right=0.1cm of T] {};
	  \node[constraint] [fill=blue,below right=0.1cm and 0.1cm of T] {};
	\end{tikzpicture}
\end{center}
\noindent
Färben wir zu erst $W$ rot, müssen die Domains von $N$ und $S$ angepasst werden, damit die Constraints wieder konsistent sind.
\noindent
\begin{center}
	\begin{tikzpicture}[auto,node distance=2.5cm,semithick, baseline=1em]
	  \tikzstyle{every state}=[fill=white,circle,draw=black,text=black,align=center]
	  \tikzstyle{constraint}=[rectangle, fill=black, below=-3pt]

	  \node[state] (N) {N};
	  \node[state] (W) [draw=red,below left=0.7cm and 1.2cm of N] {W};
	  \node[state] (S) [below of=N] {S};
	  \node[state] (T) [below right=0.7cm and 1.2cm of N] {T};

	  \path 
	  (N) edge node[constraint, label={above left:$W\neq N$},below=-3pt]{} (W)
	  edge node[constraint, label={above right:$N\neq T$}] {} (T)

	  (S) edge node[constraint, label={below right:$S\neq T$}] {} (T)
	  edge node[constraint, label={left:$N\neq S$}]{} (N)
	  edge node[constraint, label={below left:$W\neq S$}]{} (W)
	  ;

	  \node[constraint] [fill=green,above left=0.1cm and 0.01cm of N] {};
	  \node[constraint] [fill=blue,above right=0.1cm and 0.01cm of N] {};

	  \node[constraint] [fill=green,below left=0.1cm and 0.01cm of S] {};
	  \node[constraint] [fill=blue,below right=0.1cm and 0.01cm of S] {};

	  \node[constraint] [fill=red,above right=0.1cm and 0.1cm of T] {};
	  \node[constraint] [fill=green,right=0.1cm of T] {};
	  \node[constraint] [fill=blue,below right=0.1cm and 0.1cm of T] {};
	\end{tikzpicture}
\end{center}
\noindent
Legen wir nun noch $T$ auf grün fest und prüfen wiederum die Domains von $N$ und $S$ auf Konsistenz.
\noindent
\begin{center}
	\begin{tikzpicture}[auto,node distance=2.5cm,semithick, baseline=1em]
	  \tikzstyle{every state}=[fill=white,circle,draw=black,text=black,align=center]
	  \tikzstyle{constraint}=[rectangle, fill=black, below=-3pt]

	  \node[state] (N) {N};
	  \node[state] (W) [draw=red,below left=0.7cm and 1.2cm of N] {W};
	  \node[state] (S) [below of=N] {S};
	  \node[state] (T) [draw=green,below right=0.7cm and 1.2cm of N] {T};

	  \path 
	  (N) edge node[constraint, label={above left:$W\neq N$},below=-3pt]{} (W)
	  edge node[constraint, label={above right:$N\neq T$}] {} (T)

	  (S) edge node[constraint, label={below right:$S\neq T$}] {} (T)
	  edge node[constraint, label={left:$N\neq S$}]{} (N)
	  edge node[constraint, label={below left:$W\neq S$}]{} (W)
	  ;

	  \node[constraint] [fill=blue,above=0.1cm of N] {};

	  \node[constraint] [fill=blue,below=0.1cm of S] {};
	\end{tikzpicture}
\end{center}
\noindent
Man würde hier bereits nach zwei Schritten sehen, dass für $N$ und $S$ jeweils nur noch ein Wert möglich ist $D_N=D_S=\simpleset{{\color{blue}\text{blau}}}$. Aber die beiden Variablen dürfen wegen dem $N\neq S$ Constraint nicht mit dem selben Wert belegt werden.



\subsection{CSPs in Baumstruktur}
Hat der Constraintgraph eine Baumstruktur, so kann man das CSP in $\mathcal O(nd^2)$ lösen. Da dieser Aufwand so viel kleiner ist als $\mathcal O(d^n)$ eines allgemeinen CSP gibt es viele Algorithmen die versuchen ein CSP zu lösen indem es schrittweise in Baumstruktur gebracht wird.

Zunächst betrachten wir den Lösungsalgorithmus für baumstrukturierte CSP. Man wählt eine Variable als Wurzel, und ordnet alle anderen Variablen so an, dass Kindknoten immer nach ihren Eltern eingeordnet werden.
Zum Beispiel könnten die Variablen dieses Constraintgraphen folgendermaßen angeordnet werden.
\noindent
\begin{center}
	\begin{tikzpicture}[-,auto,node distance=1.5cm,
                    semithick, baseline=1em]
	  \tikzstyle{every state}=[fill=white,draw=black,text=black]

	  \node[state]         (B)  		{B};
	  \node[state] 		   (A) [above left of=B]        {A};
	  \node[state]         (C) [below left of=B] 		{C};

	  \node[state]         (D) [right of=B] 		{D};
	  \node[state]         (E) [above right of=D]       {E};
	  \node[state]         (F) [below right of=D]       {F};

	  \path (B) edge (A) edge (C) edge (D)
	  (D) edge (E) edge (F);
	\end{tikzpicture}

	\begin{tikzpicture}[->,>=stealth',shorten >=1pt,auto,node distance=1.2cm,
                    semithick, baseline=1em]
	  \tikzstyle{every state}=[fill=white,draw=black,text=black]

	  \node[state] 		   (A)         {A$_1$};
	  \node[state]         (B) [right of=A] 		{B$_2$};
	  \node[state]         (C) [right of=B] 		{C$_3$};

	  \node[state]         (D) [right of=C] 		{D$_4$};
	  \node[state]         (E) [right of=D]       {E$_5$};
	  \node[state]         (F) [right of=E]       {F$_6$};

	  \path (A) edge (B)
	  (B) edge (C) edge[in=135, out=45] (D)
	  (D) edge (E) edge[in=135, out=45] (F);
	\end{tikzpicture}
\end{center}
\noindent
Anschließend wird über diese Anordnung iteriert.
Von den Blättern zur Wurzel, d.h. für $j$ von $n$ bis $2$ werden alle Wertebelegungen aus $D_{\mathrm{Parent}(X_j)}$ gelöscht, die nicht konsistent mit dem Constraint zwischen $X_j$ und seinem Elternknoten sind.
Anschließend können von der Wurzel zu den Blättern hin Werte zugewiesen werden.
Das heißt für $j$ von $1$ bis $n$ wird $X_j$ ein Wert zugewiesen, der konsistent mit der Belegung von $\mathrm{Parent}(X_j)$ ist.
Nach der Definition von Konsistenz muss es für jede noch mögliche Belegung des Elternknotens mindestens eine zulässige Belegung von $X_j$ geben.

Wir wissen nun, dass man solche CSPs einfach lösen kann und möchten dies auf allgemeinere CSPs anwenden.

Ein baumähnliches CSP kann man \emph{konditionieren}. Dafür werden einige Variablen belegt, bis der übrige Graph ein Baum ist. Dabei müssen wir natürlich die Domains der Nachbarn der belegten Variablen entsprechend anpassen. Betrachten wir den Constraintgraphen
\noindent
\begin{center}
	\begin{tikzpicture}[auto,node distance=2cm,semithick, baseline=1em]
	  \tikzstyle{every state}=[fill=white,circle,draw=black,text=black,align=center]
	  \tikzstyle{constraint}=[rectangle, fill=black, below=-3pt]

	  \node[state] (N) {A};
	  \node[state] (S) [fill=red!50!white, below of=N] {$X$};
	  \node[state] (W) [left of=N] {B};
	  \node[state] (Q) [right of=N] {C};
	  \node[state] (E) [below of=Q] {E};
	  \node[state] (V) [below of=W] {D};

	  \path (S) edge (W) edge (N) edge (Q) edge (E) edge (V)
	  (V) edge (W)
	  (W) edge (N)
	  (N) edge (Q)
	  (Q) edge (E);
	\end{tikzpicture}
\end{center}
\noindent
Durch Entfernen von $X$ bleibt ein Baum übrig. Durch Verwenden dieses Verfahrens kann, bei $c$ \enqo{abgeschnittenen} Variablen eine Laufzeit in $\mathcal O(d^c*(d-c)d^2)$ erreicht werden, was besonders für kleine $c$ sehr gut ist.
